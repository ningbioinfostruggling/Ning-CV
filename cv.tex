%%%%%%%%%%%%%%%%%%%%%%%%%%%%%%%%%%%%%%%
% This is a modified ONE COLUMN version of
% the following template:
% 
% Deedy - One Page Two Column Resume
% LaTeX Template
% Version 1.1 (30/4/2014)
%
% Original author:
% Debarghya Das (http://debarghyadas.com)
%
% Original repository:
% https://github.com/deedydas/Deedy-Resume
%
% IMPORTANT: THIS TEMPLATE NEEDS TO BE COMPILED WITH XeLaTeX
%
% This template uses several fonts not included with Windows/Linux by
% default. If you get compilation errors saying a font is missing, find the line
% on which the font is used and either change it to a font included with your
% operating system or comment the line out to use the default font.
% 
%%%%%%%%%%%%%%%%%%%%%%%%%%%%%%%%%%%%%%
% 
% TODO:
% 1. Integrate biber/bibtex for article citation under publications.
% 2. Figure out a smoother way for the document to flow onto the next page.
% 3. Add styling information for a "Projects/Hacks" section.
% 4. Add location/address information
% 5. Merge OpenFont and MacFonts as a single sty with options.
% 
%%%%%%%%%%%%%%%%%%%%%%%%%%%%%%%%%%%%%%
%
% CHANGELOG:
% v1.1:
% 1. Fixed several compilation bugs with \renewcommand
% 2. Got Open-source fonts (Windows/Linux support)
% 3. Added Last Updated
% 4. Move Title styling into .sty
% 5. Commented .sty file.
%
%%%%%%%%%%%%%%%%%%%%%%%%%%%%%%%%%%%%%%%
%
% Known Issues:
% 1. Overflows onto second page if any column's contents are more than the
% vertical limit
% 2. Hacky space on the first bullet point on the second column.
%
%%%%%%%%%%%%%%%%%%%%%%%%%%%%%%%%%%%%%%

\documentclass[]{deedy-resume-openfont}
\usepackage{setspace}
\onehalfspacing

\begin{document}

%%%%%%%%%%%%%%%%%%%%%%%%%%%%%%%%%%%%%%
%
%     LAST UPDATED DATE
%
%%%%%%%%%%%%%%%%%%%%%%%%%%%%%%%%%%%%%%
\lastupdated

%%%%%%%%%%%%%%%%%%%%%%%%%%%%%%%%%%%%%%
%
%     TITLE NAME
%
%%%%%%%%%%%%%%%%%%%%%%%%%%%%%%%%%%%%%%


\namesection{Ning}{Liu}{ 
Ph.D in Bioinformatics (Medcine)\\
\urlstyle{same}\url{https://github.com/ningbioinfostruggling} \\
\href{mailto:liu.n@wehi.edu.au}{liu.n@wehi.edu.au} | Twitter: @lnly0311}


%\href{http://linkedin.com/in/12346}{http://linkedin.com/in/12346}

%\begin{center}
%\huge\color{subheadings}\textbf{Bioinformatician}
%\end{center}

\section{Research Expertise keywords}
\begin{flushleft}
Shell (Bash) \textbullet{} Python \textbullet{} R \textbullet{} Pipeline \textbullet{} NGS data \textbullet{} Hi-C \textbullet{} SNPs \textbullet{} Epigenetics \textbullet{} Data integration \\ 
\end{flushleft}

\section{Research Experience}

\runsubsection{Davis lab, WEHI}
%\descript{| PhD Student}
\location{May. 2021 - Present | Melbourne, Australia}
\location{Collaboration: Arutha Kulasinghe's lab in Queensland University of Technology}
Spatial Profiling of Lung and Heart SARS-CoV-2 and Influenza Virus Infection.
Main responsibilities:
\begin{tightemize}
\item Analysing spatial transcriptomics data.
\item Differential analysis.
\item Journal article writing.
\end{tightemize}
\sectionsep

\runsubsection{Davis lab, WEHI}
%\descript{| PhD Student}
\location{May. 2021 - Present | Melbourne, Australia}
Developing computation protocol to analyse spatial transcriptomics data from the NanoString GeoMX Human Whole Transcriptome Atlas platform.
Main responsibilities:
\begin{tightemize}
\item Analysing spatial transcriptomics data.
\item Normalisation method benchmarking and development.
\item Journal article writing.
\end{tightemize}
\sectionsep

\runsubsection{Jimmy Breen's lab, SAHMRI}
%\descript{| PhD Student}
\location{Feb. 2019 - Jun. 2020 | Adelaide, Australia}
\location{Collaboration: Simon Barry's lab in Women's and Children's Hospital}
Developing computation pipeline to filter T1D risk variants and investigating the potential mechanisms of the identified SNPs.
Main responsibilities:
\begin{tightemize}
\item Analysing Hi-C, ATAC-seq, ChIP-seq and SNPs data.
\item Building the filtering computational pipeline.
\item Investigating SNPs by integrating with regulatory T cell-specific epigenome data.
\item Journal article writing.
\end{tightemize}
\sectionsep

\runsubsection{Jimmy Breen's lab, SAHMRI}
%\descript{| P.hD Student}
\location{Jan. 2020 - present | Adelaide, Australia}
\location{Collaboration: ‪Hamid Alinejad-Rokny's lab in UNSW}
Identifying candidate enhancers of human cell lines and tissues from statistically signficant Hi-C interactions of public Hi-C data.
Main responsibilities:
\begin{tightemize}
\item Public Hi-C \& capture Hi-C data mining.
\item Building a computational pipeline to analyse Hi-C data based on the instruction of 4DN project.
\item Data integration and downstream analysis.
\item Journal article writing.
\end{tightemize}
\sectionsep

\runsubsection{Jimmy Breen's lab, SAHMRI}
%\descript{| P.hD Student}
\location{Jul. 2020 - Present | Adelaide, Australia}
\location{Collaboration: Sam Buckberry's lab in the University of Western Australia}
Analysing Hi-C data to identify A/B comparts and compared the genome compartmentalisation among different conditions
Main responsibilities:
\begin{tightemize}
\item Analysing Hi-C data and identifying and comparing A/B compartments under different experimental conditions.
\end{tightemize}
\sectionsep

\runsubsection{Bioinformatics hub, University of Adelaide}
%\descript{| P.hD Student}
\location{Oct. 2018 - Jan. 2019 | Adelaide, Australia}
\location{Collaboration: John Williams' lab in the Davies Research Centre, University of Adelaide}
Identifying and comparing toplogically-associated domains (TADs) between cattle Angus breed and Brahman breed.
Main responsibilities:
\begin{tightemize}
\item Analysing cattle Hi-C data and identifying TADs from Hi-C interactions.
\end{tightemize}
\sectionsep

%%%%%%%%%%%%%%%%%%%%%%%%%%%%%%%%%%%%%%
%     Publication & Award
%%%%%%%%%%%%%%%%%%%%%%%%%%%%%%%%%%%%%%

\section{Publications \& Awards}
\runsubsection{Publications}\\
\location{Journal articles}
\textbf{Liu, N.}, Sadlon, T., Wong, Y.Y., Pederson, S.M., Breen, J. and Barry, S.C., 2020. \textbf{3DFAACTS-SNP: Using regulatory T cell-specific epigenomics data to uncover candidate mechanisms of Type-1 Diabetes (T1D) risk.} bioRxiv. (peer review in progress)\\ 
\textbf{Liu, N.}, Low, W.Y., Alinejad-Rokny, H., Pederson, S., Sadlon, T., Barry, S. and Breen, J., 2020. \textbf{Seeing the forest through the trees: Identifying functional interactions from Hi-C.} bioRxiv. (peer review in progress)\\
Brown, C.Y., Sadlon, T., Hope, C.M., Wong, Y.Y., Wong, S., \textbf{Liu, N.}, Withers, H., Brown, K., Bandara, V., Gundsambuu, B. and Pederson, S., 2020. \textbf{Molecular Insights Into Regulatory T-Cell Adaptation to Self, Environment, and Host Tissues: Plasticity or Loss of Function in Autoimmune Disease. }Frontiers in Immunology, 11, p.1269.\\
Wan, Q., Leemaqz, S.Y.L., Pederson, S.M., McCullough, D., McAninch, D.C., Jankovic-Karasoulos, T., Smith, M.D., Bogias, K.J., \textbf{Liu, N.}, Breen, J. and Roberts, C.T., 2019. \textbf{Quality control measures for placental sample purity in DNA methylation array analyses.} Placenta, 88, pp.8-11.\\

\sectionsep

\location{Conference Poster}
\textbf{Ning Liu}, Timothy Sadlon, Stephen Pederson, Simon Barry \& James Breen; \textbf{Investigating computational analysis pipelines and genomic proximity interactions in T lymphocytes}; ABACBS 2017 Conference and Lorne Genome 2018 Conference; November , 2017 and February, 2018\\ 
\textbf{Ning Liu}, ‪Hamid Alinejad-Rokny, Timothy Sadlon, Stephen Pederson, Simon Barry \& James Breen; \textbf{Identifying statistically significant Hi-C interactions from regulatory T cells and development of HiC-QC}; ABACBS 2018 Conference; November, 2018\\ 
\textbf{Ning Liu}, Timothy Sadlon, Stephen Pederson, Simon Barry \& James Breen; \textbf{Using regulatory T cell-specific epigenomics data to uncover candidate mechanisms of Type-1 Diabetes (T1D)}; ABACBS 2019 Conference; November, 2019\\ 
\sectionsep

\runsubsection{Awards}\\
%\descript{| }
%\location{Adelaide, Australia}
\textbf{Best Poster Talk of COMBINE Symposium 2019} from ABACBS.\\
\textbf{Travel Grant of ABACBS Conference 2019} from ABACBS.\\
\textbf{Travel Grant of ABACBS Conference 2018} from ABACBS.\\
\textbf{Travel Grant of EMBL Australia PhD Course 2018} from EMBL Australia.\\
\textbf{Outstanding Academic Achievement Award 2015-2016} from Faculty of Sciences, University of Adelaide.\\
\textbf{Outstanding Academic Achievement Award 2016-2017} from Faculty of Sciences, University of Adelaide.\\
\sectionsep

\section{Education}
\runsubsection{SAHMRI \& Robinson Research Institute, University of Adelaide, Australia}
\descript{| PhD of Bioinformatics (Medicine)}
\location{School of Paediatrics and Reproductive Medicine,
Expected October 2020 | Adelaide, SA }
\sectionsep

\runsubsection{University of Adelaide, Australia}
\descript{| Master of Biotechnology (Biomedical)}
\location{School of Molecular and Microbiology Science,
Completed Jul 2017 | Adelaide, SA \textbullet{} GPA: 6.6/7.0}
\sectionsep

\runsubsection{Xia'men University, China}
\descript{| Bachelor of Science}
\location{Major in Immunology, Biotechnology and Genetics, and minor in C programing and biostatistics, School of Life Science | Completed Jun 2015 | Xia'men, ChinaMajor \textbullet{} GPA: 3.3/4.0}
\sectionsep

\section{Ohter Activities}
Workshop tutor of bioinforamtics workshops: Fall into bioinformatics in 2019, Spring into bioinformatics in 2019, and RAdelaide workshop 2017 \& 2018.
Workshop instructor of postgraduate course Genomics Applications (BIOINF 7150) in 2020.\\
Workshop tutor of postgraduate course Bioinformatics and Systems Modelling (BIOTECH 7005) in 2018.\\
%Contributer of R packge \textit{rGMAP}.\\
Acitve member of ABACAS since 2016.\\
Executive committee of COMBINE in 2018 \& 2019.\\
Executive committee of COMBINE SYMPOSIUM 2017 \& 2018..\\
Membership in Golden Key International Honour Society from 2015.\\

\end{document}  \documentclass[]{article}